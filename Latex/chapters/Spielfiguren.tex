\chapter{Spielfiguren}

\section{Bewegung}
Bei unseren Spielfiguren um kleine Plastiksoldaten handelt, die sich in einem kindlich dargestelltem Nah-Ostkonflikt befinden. Die Bewegung der Einheiten wird daher "uber eine Parabelkurve angedeutet, an der sich die Figur beim laufen entlang bewegt. Somit wird der Eindruck erzeugt, die Figur werde wie von einer unsichtbaren Hand in einem Brettspiel "uber das Feld bewegt.

\section{Attribute Component}


\section{Inventory Component}
Jeder einzelnen Spielfigur wird eine Inventory Component angehangen. In dieser wird das gesamte Inventar der jeweiligen Figur gespeichert. Das Inventory System k"ummert sich dabei um die Berechnungen und Aktualisierung der Inventory Komponenten.\newline
Es folgt ein Auszug der verschiedenen Variablen:\newline

\begin{lstlisting}[breaklines = true]
//Inventar (primaerwaffe, sekundaerwaffe, equipment1, equipment2) siehe Enums.cs    
public Enums.PrimaryWeapons primaryWeaponType;
public Enums.SecondaryWeapons secondaryWeaponType;
public Enums.Equipment utility1;
public Enums.Equipment utility2;

public WeaponComponent primary; //Primaerwaffe
public WeaponComponent secondary; //Sekundaerwaffe
public bool isPrimary; //Ist Primaerwaffe ausgewaehlt? 
public int amountSmokes; //Anzahl Rauchgranaten
public int amountTeargas; //Anzahl Teargas
public int amountGrenades; //Anzahl Granaten
public int amountMolotovs; //Anzahl Molotovs
public int amountMediKits; //Anzahl Medikits
public int amountMagazines; //Anzahl Magazine//Anzahl Magazine
\end{lstlisting}

\section{Selektierte Spielfigur}
Die aktuell ausgew"ahlte Spielfigur wird durch eine diese umgebende Box gekennzeichnet.
