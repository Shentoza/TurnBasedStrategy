\chapter{Fazit}
Schlussendlich lie"s sich feststellen dass die Entwicklung eines (rundenbasierten) Strategiespiels mit sehr viel Vorarbeit verbunden ist, was einen ersten Prototyp sehr verz"ogert. Es gibt sehr viel Logik zur Zugreihenfolge, der Einheiten Auswahl oder der Bewegung "uber das Spielfeld zu implementieren. Um m"oglichst schnell einen ersten spielbaren Prototyp zu haben, wurde stellenweise sehr unvorsichtig und undurchdacht implementiert. Dadurch kam es dazu, dass sehr viele Strukturen durcheinander gereicht wurden, Objekte und Variablen an verschiedenen Stellen unn"otig oft gespeichert wurden, was es Fehlern sehr leicht machte sich einzuschleichen.

Durch modulares Entwickeln beispielsweise durch Abstraktion mit Events, Listenern oder Interfaces an den entsprechenden Stellen, h"atten vielen Fehlern vorgebeugt werden k"onnen. Erw"ahnenswert ist auch der Unterschied zwischen Code der vor Ort mit den Teilnehmern entwickelt wurde, gegen"uber Code, der im ``Home Office`` entwickelt wurde. Code der in Teamsitzungen entstanden ist, machte hierbei seltenst Probleme, da sich "uber m"ogliche Schnittstellen unterhalten werden konnte, welche Variablen wo bereits vorhanden werden konnten, und allgemein ein besserer Einblick "uber den Code von anderen entstanden ist.